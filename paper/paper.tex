\documentclass[a4paper]{article}

\usepackage{xeCJK}
\setCJKmainfont[BoldFont=SimHei]{SimSun}

\usepackage[left=2.5cm,right=2.5cm,top=2.5cm,bottom=2.5cm]{geometry}
\usepackage[linesnumbered, ruled, vlined]{algorithm2e}
\usepackage{array}

\renewcommand{\baselinestretch}{1.5}

\title{基于DBLP数据库的科研合作关系挖掘}
\author{李屹(1501110045), 卢煜腾(?)}

\begin{document}
\maketitle

\section{简介}
\subsection{开发环境与运行依赖}
我们的程序在python2.7版本下开发并在Ubuntu 14.04 LTS, Ubuntu 15.10中测试通过。
由于使用了关系数据库引擎SQLite,因此运行需要依赖python中的sqlite3扩展,这个扩展可以通过\texttt{pip},
\texttt{easy\_install}等工具进行安装。如不安装这个扩展,则整个程序无法运行。

\section{数据的来源与数据库的构建}
本文所使用的数据来源于DBLP论文数据库\footnote{http://dblp.org/search/index.php}. 根据
作业要求,我们考虑了12个数据挖掘相关会议和4种相关期刊,每种会议(期刊)中选取不超过10000篇文章作为我
们的样本。实际上,由于所考虑的会议与期刊所收录总论文数均未超出该上限,因此可以认为我们抓取了上述会议
中的全部论文。

我们通过DBLP所提供的RESTful API获取论文数据,由于挖掘目标较为简单,因此我们仅靠虑文章标题,作者,年
份,期刊或会议名称等基本信息。从不同来源获得的论文数如下表所示:

\begin{figure}[ht]
  \centering
  \begin{tabular}{lll}
    \hline
    期刊/会议 & 名称(DBLP代码) & 论文数 \\
    \hline
    会议 & sdm & 1161\\
    会议 & icdm & 2274\\
    会议 & ecml-pkdd & 689\\
    会议 & pakdd & 1592\\
    会议 & wsdm & 612\\
    会议 & dmkd & 61\\
    会议 & kdd & 2575\\
    会议 & cvpr & 7049\\
    会议 & icml & 3123\\
    会议 & nips & 5599\\
    会议 & colt & 1204\\
    会议 & sigir & 3602\\
    期刊 & pattern\_recognition\_pr & 7010\\
    期刊 & sigkdd\_explorations\_sigkdd & 477\\
    期刊 & tkdd & 226\\
    期刊 & ieee\_trans\_knowl\_data\_eng\_tkde & 3101\\
    \hline
  \end{tabular}
  \label{table:dataset}
  \caption{本文所选取的期刊与会议列表}
\end{figure}

根据上述条件,我们通过json格式接口获得共40355篇论文的具体信息,并将它们整理保存在SQLite3格式的文本数
据库中。SQLite3是一种十分流行的小型关系数据库,它可以单个普通文件的形式保存,因此在处理小规模数据集
时具有方便部署的优势。同时python提供了sqlite的接口,我们的程序在运行时将通过SQL语句动态获取部分数据
。数据抓取程序在\texttt{scripts/dbgenerate.py}中实现,数据默认存储在\texttt{dataset/default.sqlite}
中。如果不需要重新获取数据则不需要再次运行这个脚本。

在这次作业中,我们抓取的数据字段包括:标题,作者,年份,期刊或会议名称。

\section{频繁项集挖掘算法与优化}

\section{基于先验知识的合作关系分析}
在本次作业的数据挖掘过程中,我们仅仅通过FP-Growth算法挖掘了频繁项集,而没有进行进一步的关系挖掘。之
所以这样做,主要是考虑到各项之间的关系在时间上具有不一致性。举例来说,``如果一个人买了啤酒,那么他也
有很大的概率会买尿布''这条性质将会在很长的一段时间内保持稳定,然而``如果教授A发表了一篇文章,那么这
篇文章的合作者中很大几率会出现教授B''这种关系却随着时间不断变化。

我们假设有研究人员A、B、C、D,两个人十年之前曾经紧密合作并发表了10篇论文,而之后分别又发表了90篇。而
C与D均刚开始工作,两人共同合作发表了5篇论文,此外并没有单独发表任何文章。我们很难断言C与D的合作关系
要比A与B的关系频繁,然而如果通过关系分析便很有可能得到这样的结论。

基于这样的考虑,后续的合作关系分析主要基于对学术领域的先验知识。这部分中描述的所有算法均在
\texttt{lib/analysis.py}中实现。

\subsection{导师-学生指导关系和指 导时间}
一般而言,一个比较典型的学生-导师的关系应当包括下述特征:
\begin{enumerate}
  \item 两人在一段时期内频繁合作发表文章
  \item 这样的合作关系通常时间不短于一年
  \item 导师与学生在学术领域的积累通常有较大的稳定差距
  \item 一个导师通常指导多个学生
\end{enumerate}

在实际的算法中,我们通过下述的筛选过程来寻找导师-学生指导关系:首先选取FP-Growth中计算得到的所有频繁
2项集, 并考察每个频繁项中的两个作者。如果这两个作者首次发表论文的时间差不小于某个阈值
(本文中设定的阈值为8)\footnote{根据多位老师同学的意见,10年是一个更为合适的阈值,但是在特定情况下
(老师开发表文章较晚而收学生较早)部分关系可能被错误地过滤掉。因此我们设定了较低的阈值并追加进行后续
判断, 希望借此得到更加准确的结果},且资历较老的研究人员发表论文数目明显较多(这里设定为超出5篇或以上
)那么我们初步认为这是一对学生-导师的关系。

接下来我们对这些关系按导师归类,如果归类后一个导师旗下有超过两名学生,那么认定这是一组合理的导师-学
生指导关系。反之,则认为指导关系较弱,将其从指导关系中删除。通过上述的算法,我们挖掘到了较为可靠的导
师-学生指导关系,具体结果将在下一章中详细描述。

如果仅靠虑国内高校或研究机构的成果,其实可以考虑作者的顺序。通常而言导师署名排在最末并注明为通讯作者
,而学生往往排在第一作者的位置。然而在国外,许多研究人员倾向使用字母序,因此这种方法我们并未采用。

\subsection{频繁合作关系与合作团队}
在本次作业中, 我们基于高频高阶频繁项来进行频繁合作关系与合作团队的挖掘。

字面上看来,
\begin{figure}[ht]
  \begin{center}
    \begin{tabular}{m{12cm}r}
      \hline
      团队成员 & 合作指数 \\
      \hline
      Shiguang Shan, Ruiping Wang, Xilin Chen, Zhiwu Huang & 12 \\
      Dinh Q. Phung, Svetha Venkatesh, Sunil Kumar Gupta, Duc-Son Pham, Budhaditya Saha, Santu Rana
      & 16 \\
      Huan Liu, Jiliang Tang, Xia Hu, Huiji Gao & 10 \\
      Shinjae Yoo, Hong Qin, Dantong Yu, Hao Huang & 12 \\
      Sethuraman Panchanathan, Wei Fan, Ian Davidson, Jieping Ye & 10 \\
      Jing Gao, Kang Li, Nan Du, Aidong Zhang & 8 \\
      Jiafeng Guo, Shuzi Niu, Yanyan Lan, Xueqi Cheng & 8 \\
      \hline
    \end{tabular}
  \end{center}
  \caption{频繁合作的团队}
\end{figure}

\section{关系验证}
根据作业要求,我们对挖掘出来的导师-学生指导关系进行了人工验证。在此处我们所定义的导师-学生
指导关系包括:Master, PhD, PostDoc以及其他合作指导关系(由学生在页面上明确说明的)。

\begin{figure}[ht]
  \begin{center}
    \begin{tabular}{lm{9cm}l}
      \hline
      导师 & 学生 & 准确率 \\
      \hline
      Ron Kohavi & Toby Walker , Ya Xu  & 0\%\\
      Nicolò Cesa-Bianchi & Fabio Vitale , Giovanni Zappella  & 100\%\\
      Maarten de Rijke & Wouter Weerkamp , Ilya Markov  & 100\%\\
      Shiguang Shan & Zhiwu Huang , Ruiping Wang  & 100\%\\
      Claudio Gentile & Fabio Vitale , Giovanni Zappella  & 100\%\\
      Michael I. Jordan & Fabian L. Wauthier , Martin J. Wainwright  & 100\% \\
      Huan Liu & Xia Hu , Huiji Gao  & 100\%\\
      Christos Faloutsos & Partha Pratim Talukdar , Shiqiang Yang , Alex Beutel , Meng Jiang , Nicholas D.
      Sidiropoulos  & 20\%\\
      Philip S. Yu & Philippe Fournier-Viger , Chang-Dong Wang , Bo Liu , Cheng-Wei Wu , Zhifeng Hao ,
      Chaokun Wang , Bokai Cao , Jun Zhang , Hong-Han Shuai , Yanshan Xiao  & 0\%\\
      Xingquan Zhu & Shirui Pan , Li Guo , Jia Wu  & \\
      Jiawei Han & Latifur Khan , Xiang Ren , Deng Cai , Xifeng Yan , Jing Gao , Mohammad M. Masud , Xiao
      Yu  & 57.14\%\\
      Wei Fan & ErHeng Zhong , Sethuraman Panchanathan  & 50\%\\
      \hline
    \end{tabular}
  \end{center}
  \caption{导师-学生指导关系的准确率}
\end{figure}

经过人工核验,我们找出的大多数师生关系均正确,其中准确率较低的各关系也能够得到解释:
\begin{description}
  \item[Ron Kohavi] 是
    微软研究院的一名研究人员。由于微软研究院给出的页面上并不包括学生信息,因此我们猜想他的``学生''实
    际上应当是研究院中Mentor与实习生或者下属的关系,但是这一层无法得到验证。
  \item[Jiawei Han] 与\textbf{Philip S. Yu}, \textbf{Christos Faloutsos}等人,由于在业界工作时间过长
    ,已经与很多人建立了疑似指导但并非名义师生的关系,这会造成误判。
\end{description}

\section{小结与分工}

\end{document}
